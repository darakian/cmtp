\documentclass[a4paper,11pt]{article}
\usepackage[T1]{fontenc}
\usepackage[utf8]{inputenc}
\usepackage{lmodern}
\usepackage{url}

\title{The Cipher Mail Transport Protocol Cryptography}
\author{Jonathan Moroney}

\begin{document}

\maketitle
%\tableofcontents

%\begin{abstract}
%\end{abstract}

\section{Overview}
The Cipher Mail Transport Protocol (CMTP) has been designed to easily adapt to new crypto-systems as the need arises. Each communication from a server has a version field which denotes the crypto-system in use. There are two currently defined crypto-systems:
\begin{enumerate}
  \item 0 - Plaintext. That is, no encryption.
  \item 1 - X25519 / Salsa20. Further details below.
\end{enumerate}
As the state of cryptography evolves additional crypto systems can be added.

\section{Crypto system 1}
The choice for crypto-system 1 to use X25519 / Salsa20 and the signing and securing algorithms was a fundamentally pragmatic one. In particular messages are enciphered with the LibSodium crypto\_box\_seal() \cite{crypto-box-seal} function. This function does the job of encrypting and signing the message as well as throwing errors should something not match up. The signing function is the crypto\_sign() function \cite{crypto-sign}. Many large players on the internet are using this library and these algorithms which helped make this decision easier to make. However, this does end up meaning that the ``correct" implementation of encryption is whatever LibSodium does and I don't feel good in punting like that.


\bibliographystyle{acm}
\bibliography{docs-references}
\end{document}
