\documentclass[a4paper,11pt]{article}
\usepackage[T1]{fontenc}
\usepackage[utf8]{inputenc}
\usepackage{lmodern}

\title{The Cipher Mail Transport Protocol Interface}
\author{Jonathan Moroney}

\begin{document}

\maketitle
\tableofcontents

%\begin{abstract}
%\end{abstract}

\section{High level overview}
The Cipher Mail Transport Protocol (CMTP) uses a text based, command/reply interface similar to that of the Simple Mail Transport Protocol (SMTP). The commands for CMTP have been designed to make stateless implementation easy. All commands are terminated by a line feed (\textbackslash n).
\begin{description}
  \item [OHAI] \hfill \\
  This command is used to identify a CMTP client to a CMTP server. This is the CMTP analog to SMTPs HELO/EHLO though there are no parameters that follow.
  \item [MAIL] \hfill \\
  This command is used to initiate a mail transfer. After the command is issued the CMTP server should expect a self describing CMTP message to follow and should reply after it has been received.
  \item [KEYREQUEST <USER> <DOMAIN>] \hfill \\
  This command is used to request a public key for some user on some domain. The parameters USER and DOMAIN are null terminated.
  \item [NOOP] \hfill \\
  This command does nothing but prompt a reply from the CMTP server.
  \item [LOGIN <USER>] \hfill \\
  This command is used to initiate a user login.
  \item [HELP] \hfill \\
  This command is used by users that need to RTFM.
  \item [OBAI] \hfill \\
  This command is used to terminate a connection.
\end{description}

\end{document}
