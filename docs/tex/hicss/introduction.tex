\section{Introduction}
Email is an integral part of modern communications and daily life. It's used for nearly all formal communications and it's a broken system. Email has evolved from the ancient times to provide an asynchronous messaging system where users are tied to servers which facilitate mail and servers are tied to each other to ensure that they can facilitate mail. This relationship between the user and their server is something that allows mail to work, but at the same time is something that can be revised. For example; a generic user will trust their server to (attempt to) deliver any mail they send to it, they will trust it to receive mail on their behalf and to occasionally forward their mail to another server. To the user a mail server is a diligent butler. However, as has become evidently clear in the past few years, users are trusting their infrastructure far more than is reasonable.
\paragraph{}
The infrastructure in question dates back to 1982 when SMTP was first defined and stretches to the present day with ESMTP, POP, IMAP, OpenPGP, and webmail\cite{rfc821}. This infrastructure includes many disparate servers across the globe talking the same language and making use of other internet infrastructure such as the DNS system. I love this infrastructure and think there is a great merit in allowing any idiot out there to become a node, but this is also where the trust issues stem from as some idiot might not be so idiotic and might be quite hostile. As such, it is time for a new protocol that acknowledges the hostility of the modern network.
\subsection{Modern Email}
For the purposes of this paper ``modern email'' will refer to SMTP used in conjunction with IMAP or a webmail client. POP, while historically important, is not considered to be modern. Modern email is a system with which users can send and receive email from many devices/programs, can change programs/devices without administrative approval, and can deal with email in an asynchronous fashion. It's important to understand that email is a generic term for a number of different systems and that by modern email; I mean none of them specifically, but all of them generally. That is, all of their ``good'' parts generally. For instance, SMTP originally required every user to have an SMTP server of their own and this I qualify as bad. Conversely, gmail allows a user to travel the globe and to have full access to their communications from anywhere so long as they know some login information; this I qualify as good. It's unfortunate that the good quality of email that I bring up can be exploited if you are: in the wrong country, are the wrong person, cross the wrong IT guy, are at the wrong coffee shop, etc...
\subsection{Motivation}
This section may be of a very different tone from the others, but I wanted to include it so that my impetus is clear. Encryption has been an interest of mine for about as long as I've been aware of it, and secure messaging the most interesting application. I used to think that encryption worked quite well until I actually had to use it in \textit{``the real world''}
\begin{lstlisting}
A few years ago, I had the opportunity to be an intern at a large tech trade show (Interop) and it was a great thing to have done. However, in order to get reimbursed for hotel, flight, and the like I had to get in contact with a woman in New York. She needed to fill out some forms with information that I'd rather not be in plaintext as it traveled a quarter of the way around the globe. So, I encrypted it in a password protected zip and sent it on its way. What followed was a week of this woman being pissed at me for making her life that little bit harder and some awkward social situations that I had to navigate while at the show.
\end{lstlisting}
It's probably not a unique story and it's easy to just place blame for the situation and move on. But, it's the triviality of it that stuck with me and it's the infrastructure that was at fault. Couple that with being in grad school and needing a thesis project and we arrive at this document.
